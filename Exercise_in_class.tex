\documentclass[10pt]{article}
\usepackage{pictex,graphicx,amsmath,amssymb,amsbsy,amsfonts,amsthm,verbatim}
\usepackage{graphics}
\usepackage{fullpage}
\usepackage{fancyhdr}
\usepackage{algorithm,algorithmic}
\usepackage{multirow}
\setlength{\voffset}{-0,25in}
\setlength{\headsep}{+0,5in}
\setlength{\parskip}{1em}
\setlength{\parindent}{0em}

\newcounter{problem}
\newcommand{\problem}{\textbf{\refstepcounter{problem}Problem \theproblem} }

\def\vu{\mathbf{u}}
\def\vs{\mathbf{x}}
\def\vb{\mathbf{b}}
\def\vv{\mathbf{v}}
\def\vw{\mathbf{w}}
\renewcommand{\implies}{\rightarrow}
\renewcommand{\lor}{\vee}
\renewcommand{\land}{wedge}
\renewcommand{\iff}{\lefftrightarrow}
\newcommand{\xor}{\oplus}
\newcommand{\TRUE}{\mathbf{T}}
\newcommand{\FALSE}{\mathbf{F}}
\newcommand{\universe}{\mathcal{U}}

\begin{document}
\pagestyle{fancyplain}
%%%%%%%%%%%%%%%%%%%%%%%%%%%%%%%%
\chead{DICRETE STRUCTURE (CO1007) --- Homework 03 --- PROOF}
\textit{\textbf}{\underline{Instruction:}} Type your answer to the following questions provided by Latex and submit a zipped file ( included .pdf file and .tex file) to E-learning by group (only 4-5 member in each group). Only team leader will submit it. One page per problem. Please use the solution template provided)

\begin{center}
  \begin{tabular}{|c|c|c|c|}
     \hline
      \multicolumn{4}{|c|}{\textbf{GROUP ... ----MEMBER LIST}}\\
      \hline
      \textbf{No.} &\qquad\qquad \textbf{Name}\qquad\qquad\qquad & \qquad\qquad{ID}\qquad\qquad & \qquad{Role}\qquad\qquad \\
      \hline
        1 & Pham Minh Tuan & 1752595 & \\
      \hline
\end{tabular}
\end{center}
\bigbreak
\problem Show that: P(n)
       1+2+3+ .... + n = $\dfrac{n \times (n+1)}{2}$
\bigbreak
\textit{Solution:}\\
\textbf{Basic step:} $With n = 1 \rightarrow P(1) is true$\\ 
\textbf{Inductive step:}\\ 
     As n = k, P(k) = 1+2+3+ ...+k = $\dfrac{k \times (k+1)}{2}$
\bigbreak
     As n = k+1, P(k+1) = 1+2+3+ ... + (k+1) =  $\dfrac{(k+1) \times ((k+1)+1)}{2}$\\
     $\rightarrow P(k+1) is true$\\
     $\rightarrow P(k) is true$\\
     $\therefore P(n) is true$
\clearpage
\problem Prove that P(n): $3^{2n-1}$ + 1 \vdots 4 for all $n \ge 1$
\bigbreak
\textbf{Basic step:}
\bigbreak
$3^{1 \times 2 -1}$ + 1 = 4 \vdots 4 (n =1)
\bigbreak
\textbf{Inductive step:}
\bigbreak
\begin{enumerate}
%1
\item n = k :\\	
    $3^{2k -1}$ +1 $\vdots 4$ \\
 %2
\item n = k +1 :\\
    $3^{2k + 1} +1 \vdots 4$ \\
$\leftrightarrow 3^{2k -1 +2} +1 \vdots 4$ \\
$\leftrightarrow 3^{2} \times 3^{2k - 1} +1 -3^{2} +3^{2} \vdots 4$\\
$\leftrightarrow 3^{2} \times (3^{2k -1} + 1) + (-8) \vdots 4$\\
     $\rightarrow P(k+1) \vdots 4 $\\
     $\rightarrow P(k) \vdots 4$\\
     $\therefore P(n) \vdots 4$
\end{enumerate}
\clearpage
\problem Prove that P(n): $6^{n} -1 \vdots 5, \forall n \ge 1$
\bigbreak
\textbf{Solution:}
\bigbreak
\textbf{Basic step}\\
    $6^{1} - 1 \vdots 5$ (for n=1)\\
\textbf{Inductive step}
\begin{enumerate}
%1 
\item n = k\\
    $6^{k} - 1 \vdots 5$\\
%2
\item n = k+1\\
    $6^{k +1} - 1 \vdots 5$\\
$\leftrightarrow 6 \times 6^{k} -6 + 6 \vdots 5$\\
$\leftrightarrow 6 \times (6^{k} -1) + 5 \vdots 5$\\
     $\rightarrow P(k+1) \vdots 5 $\\
     $\rightarrow P(k) \vdots 5$\\
     $\therefore P(n) \vdots 5$
\end{enumerate}
\clearpage
\problem $F_{0}$; $F_{1}$; $F_{n+2}$ = $F_{n}$ + $F_{n+1}$ for $n \ge )$. Prove that $F_{3n}$ is even for $n \ge 1$\\
\textbf{Solution:}
\bigbreak
\textbf{Basic step}\\
$F_3$ = $F_1$ + $F_2$ = 1 + 1 = 2\\
$\rightarrow F_3$ is even (n = 1)\\
\begin{enumerate}
%1 
\item n= k\\
$F_{k+2}$ = $F_{k}$ + $F_{k+1}$\\
$\rightarrow F_{3k}$ is even
%2
\item n = k+1\\
$F_{3k+3}$ = $F_{3k+1}$ + $F_{3k+2}$\\
\qquad\qquad =$F_{3k} + 2\times F_{3k+1}$\\
$\rightarrow F_{3k +3}$ is even\\
$\therefore F_{3n}$ is even
\end{enumerate}
\clearpage
\problem Show that if n is integer and $n^{3}$ + 2015 is odd, then n is even\\ 
a) Contraposition:\\
\textbf{Solution:}\\
We assume n is odd : n = 2k +1\\
\qquad\qquad\qquad $n^{3}$ = $(2k +1)^{3}$\\
We have: 
   $n^{3}$ + 2015 \\
 = 8$k^{3}$ + 12$k^{2}$ + 6k + 1 + 2015\\
 = 8$k^{3}$ + 12$k^{2}$ + 6k + 2016\\ 
 = $2 \times (4k^{3} + 6k^{2} + 3k + 1008)$\\
$\rightarrow n^{3} + 2015$ is even\\
So if $n^{3}$ + 2015 is odd then n is even.
\clearpage
\problem Prove that sum of two rational number is a rational number\\
\textbf{Solution:}\\
We assume that: $\dfrac{a}{b}$ and $\dfrac{c}{d}$ is two rational number\\
$\dfrac{a}{b}$ + $\dfrac{c}{d}$ = $\dfrac{ad-bc}{bd}$\\
As x = $\dfrac{a}{b} and y = \dfrac{c}{d}$\\
We assume that x + y is rational number
 $\rightarrow Q.E.D.$
\end{document}




