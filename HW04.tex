\documentclass[10pt]{article}
\usepackage{pictex,graphicx,amsmath,amssymb,amsbsy,amsfonts,amsthm,verbatim}
\usepackage{graphics}
\usepackage{fullpage}
\usepackage{fancyhdr}
\usepackage{algorithm,algorithmic}
\usepackage{multirow}
\setlength{\voffset}{-0,25in}
\setlength{\headsep}{+0,5in}
\setlength{\parskip}{1em}
\setlength{\parindent}{0em}

\newcounter{problem}
\newcommand{\problem}{\textbf{\refstepcounter{problem}Problem \theproblem} }

\def\vu{\mathbf{u}}
\def\vs{\mathbf{x}}
\def\vb{\mathbf{b}}
\def\vv{mathbf{v}}
\def\vw{\mathbf{w}}
\renewcommand{\implies}{\rightarrow}
\renewcommand{\lor}{\vee}
\renewcommand{\land}{wedge}
\renewcommand{\iff}{\leffrightarrow}
\newcommand{\xor}{\oplus}
\newcommand{\TRUE}{\mathbf{T}}
\newcommand{\FALSE}{\mathbf{F}}
\newcommand{\universe}{\mathcal{U}}

\begin{document}
\pagestyle{fancyplain}
%%%%%%%%%%%%%%%%%%%%%%%%%%%
\chead{DICRETE STRUCTURE (CC007) --- HOMEWORK 04 --- SET \& FUNCTION}
\textit{\textbf}{\underline{Instruction:}} Type your answer to the following question provided by Latex and submit a zipped file (included .pdf file and .tex file) to E-learning by group (only 4-5 member in each group). Only team leader will submit it. One page per problem. Please use the solution template provided)

\begin{center}
 \begin{tabular}{|c|c|c|c|}
  \hline
  \multicolumn{4}{|c|}{\textbf{GROUP ... ----MEMBER LIST}}\\
  \hline
  \textbf{No.} &\qquad\qquad \textbf{Name}\qquad\qquad\qquad & \qquad\qquad{ID} \qquad\qquad & \qquad{Role} \qquad\qquad \\
   \hline
   1 & Pham Minh Tuan & 1752595 & \\
   \hline
\end{tabular}
\end{center}

\problem [5pt] Prove the following statements
\begin{enumerate}
%1
\item $\overline{A \cap B \cap C} = \overline{A} \cup \overline{B} \cup \overline{C}$\\
 \textbf{1)}Suppose that:$ x \in \overline{A \cap B \cap C}$\\
\qquad\qquad $\leftrightarrow x \in \overline{A} \cap x \in \overline{B} \cap x \in \overline{C}$\\
\qquad\qquad $\leftrightarrow x \notin A \cap x \notin B \cap x \notin C$\\
\qquad\qquad $\leftrightarrow x \in \neg{A \cap B \cap C}$\\
\qquad\qquad $\leftrightarrow x \in (\overline{A} \cup \overline{B} \cup \overline{C})$\\
$\rightarrow Q.E.D$\\
\textbf{2)} Membership Table:
\begin{center}
   \begin{tabular}{|c|c|c|c|c|}
   \hline
   \textbf{A} & \textbf{B} & \textbf{C} & \qquad \textbf{$\overline{A \cap B \cap C}$} &\qquad \textbf{$\overline{A} \cup \overline{B} \cup \overline{C}$}\\
    \hline
    1 & 1 & 1 & 0 & 0\\
    0 & 0 & 0 &1 & 1\\
    0 & 1 & 1 & 0 & 0\\
    0 & 1 & 0 & 0 & 0\\
    1 & 0 & 0 & 0 & 0\\
    1 & 0 & 1 & 0 & 0\\
    0 & 0 & 1 & 0 & 0\\
    1 & 1 & 0 & 0 & 0\\
     \hline
    \end{tabular}
\end{center}
%2
\item $P(A) \subseteq P(B)  \mbox{ if and only if} A \subseteq B$\\
Suppose that $A \nsubseteq B \mbox{ and}  A \in P(A), B \in P(B)$\\
$\rightarrow P(A) \nsubseteq P(B) \mbox{so if} A \subseteq B \mbox{then} P(A) \subseteq P(B)$
%3
\item $(B - A) \cup (C -A) = (B \cup C) - A$\\
Suppose that:
A= $\left\{1,2,3\right\}$\\
B= $\left\{3,4,5\right\}$\\
C= $\left\{2,7,8\right\}$\\
We have: B - A = $\left\{4,5\right\}$\\ 
              C - A = $\left\{7,8\right\}$\\
$\rightarrow (B - A) \cup (C - A) = \left\{4,5,7,8\right\}$ (1)\\
We have: $B \cup C = \left\{2,3,4,5,7,8\right\}$\\
$\rightarrow (B \cup C) - A = \left\{4,5,7,8\right\}$ (2)\\
From (1), (2) $\rightarrow Q.E.D$         
%4
\item $\mbox{Explain why } A \times B \times C \mbox{ and }  (A \times B) \times C \mbox{ are not the same.}$\\
Suppose that:\\
A= $\left\{1\right\}$\\
B= $\left\{3,4\right\}$\\
C= $\left\{5,6\right\}$\\ 
We have $A \times B \times C = \left\{(1,3,5),(1,3,6),(1,4,5),(1,4,6)\right\}$ (1) \\
We also have $(A \times B) = \left\{(1,3),(1,4)\right\}$\\
$\rightarrow (A \times B) \times C = \left\{(1,5),(1,6),(3,5),(3,6),(4,5),(4,6)\right\}$ (2) \\
$\therefore \mbox{ (1) and (2) are different }$\\
$\rightarrow Q.E.D$
\end{enumerate}
\clearpage
\problem [5pt] The symmetric differrence of A and B, denoted by $A \oplus B$, is the set containing those elements in either A or B, but not in both A and B.
\begin{enumerate}
%1
\item $ \mbox{ Show that} A \oplus B = (A \cup B)(A \cap B)$\\
We define that $A \oplus B =(A - B) \cup (B - A)$\\
Suppose that:\\
A= $\left\{1,2,3\right\}$\\
B= $\left\{3,4,5\right\}$\\
We have: A - B =$\left\{1,2\right\}$\\
              B - A =$\left\{4,5\right\}$\\
$\rightarrow (A - B) \cup (B - A) = \left\{1,2,4,5\right\}$ (1) \\
Beside:$(A \cup B)(A \cap B) = \left\{1,2,4,5\right\}$ (2) \\
$(1) \& (2) \therefore  A \oplus B = (A \cup B)(A \cap B)$
%2
\item $ \mbox{ What can you say about the sets A and B if} A \oplus B = A?$\\
A is the set and B is the empty set or B is the set and A is the empty set\\
Definitely: $A \oplus B = (A - B) \cup (B - A)$\\
So if $B = \emptyset \mbox{ then } A - B = A - \emptyset = A$\\
       $B - A = \emptyset - A = \emptyset $\\
$\rightarrow (A - B) \cup (B - A) = A \cup \emptyset = A$\\
$\therefore Q.E.D$
%3 
\item $ \mbox{ If A, B, C are sets, does it follow that} A \oplus (B \oplus C) = (A \oplus B) \oplus C?$\\
Use Membership Table:
\begin{center}
   \begin{tabular}{|c|c|c|c|c|}
    \hline
    A B C & \quad $B \oplus C$ & \quad $A \oplus B$ & \quad $A \oplus (B \oplus C)$ & \quad $(A \oplus B) \oplus C$\\
     \hline
     1 1 1 & 0 & 0 & 1 &1\\
     1 1 0 & 1 & 0 & 0 &0\\
     1 0 1 & 1 & 1 & 0 &0\\
     1 0 0 & 0 & 1 & 1 &1\\
     0 1 1 & 0 & 1 & 0 &0\\
     0 1 0 & 1 & 1 & 1 &1\\
     0 0 1 & 1 & 0 & 1 &1\\
     0 0 0 & 0 & 0 & 0 &0\\
     \hline
    \end{tabular}
\end{center}
\end{enumerate}
\clearpage
\problem [5pt] What can you say about the sets A and B if we know that 
\begin{enumerate}
%1
\item $A \cup B = A?$\\
We assume that B is the empty set\\
As $B = \emptyset  \mbox{ so } A \cup B = A$ 
%2
\item $A \cap B = A?$\\
B can be the subset of A, denoted by $ B \subseteq A$\\
$\rightarrow A \cap B = A$ 
%3
\item $A - B = A?$\\
B is the empty set or $B \cap A = \emptyset$\\
%4
\item $A \cap B = B \cap A?$\\
when $A \subseteq B \mbox{ and } B \subseteq A$\\
$\left\{x|x \in A \mbox{ and }x \in B\right\}$
%5
\item $A - B = B - A?$\\
When A = B, as $A \subseteq B \mbox{ and } B \subseteq A$\\
\end{enumerate}
\clearpage
\problem [5pt] Find the domain and range of these functions. Note that in each case, find the domain, determine the set of set of elements assigned values by the function.
\begin{enumerate}
%1
\item The function that assigns to each bit string the number of ones in the string minus the number of zeros in the string\\
Domain: The set of all bit strings.\\
Range:$\mathbb{Z}$
%2
\item The function that assigns to each bit string twice the number of zeros in the string.\\
Domain: The set of all bit strings.\\
Range:$R = \left\{x,n \in \mathbb{N^{*}}|x = 2n\right\}$\\
%3
\item The function that assigns the number of bits left over when a bit string is split into bytes (which are blocks of 8 bits)\\
Domain: The set of bit strings.\\
Range:$R = \left\{0,1,2,3,4,5,6,7\right\}$\\
%4
\item The function that assigns to each possitve integer the largest perfect square not exceeding this integer\\
Domain: The set of positive integer.\\
Range: $R = \left\{1,4,9,16,25...\right\}$\\
\end{enumerate}
\clearpage
\problem [5pt] Determine whether each of these functions is a bijection from \textbf{R} to \textbf{R}
\begin{enumerate}
%1
\item f(x) = -3x + 4.\\
Suppose that: -3x + 4 = -3y + 4 $\leftrightarrow{-3x = -3y}$ $\leftrightarrow{x = y}$\\
$\rightarrow{\mbox{ The function is injective }}$\\
We have: y = -3x +4, so x = $\dfrac{y - 4}{-3}$\\
$\therefore \exists x = \dfrac{y - 4}{-3}$ that f(x) = y.\\
$\rightarrow{\mbox{ The function is surjective. }}$\\
$\therefore{\mbox{ The function is injection. }}$
%2
\item f(x) = -3$x^{2}$ +7.\\
Suppose that: -3$x^{2}$ +7 = -3$y^{2}$ +7 $\leftrightarrow{x^2 = y^2}$ $\leftrightarrow{x = y \mbox{ or } x = -y}$\\
$\rightarrow{\mbox{ The function is not injective. }}$\\
$\therefore{\mbox{ The function is not a bijection. }}$\\
%3
\item f(x) = $\dfrac{x + 1}{x + 2}$\\
Suppose that: $\dfrac{x + 1}{x + 2} = \dfrac{x +1}{x +2}$\\ 
$\leftrightarrow{(x + 1)(y +2) = (y + 1)(x + 2)}$\\
 $\leftrightarrow{xy + x + 2y + 2 = xy + 2x + y + 2}$\\
  $\leftrightarrow{x = y}$\\
We have y = $\dfrac{x + 1}{x + 2}$\\
$\leftrightarrow{y = \dfrac{x + 2- 1}{x + 2}}$\\
$\leftrightarrow{ y = 1 - \dfrac{1}{x + 2}}$\\
$\leftrightarrow{ 1 - y = \dfrac{1}{x + 2} }$\\
$\leftrightarrow{ x + 2 = \dfrac{1}{1 - y}}$\\
$\leftrightarrow{ x = \dfrac{1}{1 - y} - 2}$\\
So, $\forall y \exists x = \dfrac{1}{1 - y} - 2$\\
$\rightarrow{\mbox{ The function is surjective }}$\\
$\therefore{\mbox{ The function is bijective }}$  
%4
\item f(x) = $x^5 + 1$\\
Suppose that: $x^5 + 1 = y^5 + 1$\\
$\leftrightarrow{x^5 = y^5}$\\
$\therefore{x = y}$\\
$\rightarrow{\mbox{ The function is injective }}$\\
We have: $y = x^5 + 1$\\
$\leftrightarrow{x = \sqrt[5]{y - 1}}$\\
$\rightarrow \forall y \exists x = \sqrt[5]{y - 1}$\\
$\rightarrow{\mbox{ The function is surjective }}$\\
$\therefore{\mbox{ The function is a bijection }}$\\ 
%5
\item f(x) = $\dfrac{x^{5} +1}{x^{2} +2}$\\
Suppose that: $\dfrac{x^{5} +1}{x^{2} +2} = \dfrac{y^{5} +1}{y^{2} +2}$\\
$\leftrightarrow{(x^5 + 1)(y^2 +2) = (y^5 + 1)(x^2 + 2)}$\\
 $\leftrightarrow{x^5y^2 + 2x^5 + y^2 + 2 = y^5x^2 + 2y^5 + x^2 + 2}$\\
  $\leftrightarrow{x = y}$\\
We have: y = $\dfrac{x^{5} +1}{x^{2} +2}$     
\end{enumerate}
\end{document}